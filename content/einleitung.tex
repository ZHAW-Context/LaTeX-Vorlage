%!TEX root = ../doc.tex
\chapter{Einleitung}
\label{sec:Einleitung}

\section{Ausgangslage}
\label{sec:Ausgangslage}

\textcolor{darkgray}{
  \begin{itemize}
  \item Nennt bestehende Arbeiten/Literatur zum Thema -> Literaturrecherche
  \item Stand der Technik: Bisherige Lösungen des Problems und deren Grenzen
  \item (Nennt kurz den Industriepartner und/oder weitere Kooperationspartner und dessen/deren Interesse am Thema Fragestellung)
  \end{itemize}
}

\section{Zielsetzung / Aufgabenstellung / Anforderungen}
\label{sec:ZielsetzungAufgabenstellungAnforderungen}

\textcolor{darkgray}{
  \begin{itemize}
  \item Formuliert das Ziel der Arbeit
  \item Verweist auf die offizielle Aufgabenstellung des/der Dozierenden im Anhang
  \item (Pflichtenheft, Spezifikation)
  \item (Spezifiziert die Anforderungen an das Resultat der Arbeit)
  \item (Übersicht über die Arbeit: stellt die folgenden Teile der Arbeit kurz vor)
  \item (Angaben zum Zielpublikum: nennt das für die Arbeit vorausgesetzte Wissen)
  \item (Terminologie: Definiert die in der Arbeit verwendeten Begriffe)
  \end{itemize}
}